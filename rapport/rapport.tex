\documentclass[a4paper]{article}
\usepackage[utf8]{inputenc}
\usepackage{xspace}
\usepackage[french]{babel}
%\usepackage{amsmath}
\usepackage{xcolor}
\title{Rapport}
\author{ Handy-Pedro \textsc{Valery}\\
  %(valeryhandypedro@yahoo.com)
  Charles \textsc{Dehlinger}\\
  %(charles.dehlinger@universite-paris-saclay.fr)
}
\date{Dimanche 14 juin 2020}
\setcounter{tocdepth}{4}

\begin{document}
\maketitle
\tableofcontents

\section{introduction}
Dans ce projet, nous avons été amemés à chercher la meilleur façon de
détecter la langue maternelle d'un locuteur lorsqu'il s'exprime en
anglais.\\
Nous avons choisit de regarder les mots typiques utilisés par
certains locuteurs. En effet certains mots sont transparants dans
certaines langues et sont donc plus utilisés par ceux qui la
pratiquent.

\section{le traitement des données}
Les données sont traitées de la façon suivante :
\begin{itemize}
\item On sépare les phrases, selon les points (.),
\item On découpe chaque phrase en mots grâce à une expression
  régulière,
\item On obtient un tableau de phrases, chaque phrase étant un tableau
  de mots.
\end{itemize}
À la suite de celà, on peut travailler phrase par phrase.
\section{analyse des données}
\subsection{les mots}
Nous avons d'abord essayé un procédé fondé sur de l'inférence
bayésienne. On commence par calculer la probabilité qu'un mot se
trouve dans un texte d'une langue donnée $P_l(m)$, puis la probabilité
qu'un texte soit de cette langue $P(l)$ (uniforme dans le cas
présent), en suite on calcule la probabilité que ce mot se trouve dans
un texte quelconque $P(m)$, enfin on peut estimer la probabilité qu'un
texte soit dans un texte sachant qu'un mot s'y trouve $P_m(l)$, grâce
à la formule de bayes :
$$P_m(l) = \frac{P_l(m) \cdot P(l)}{P(m)}$$

Malheureusement ce procédé nous a donné des résultats désastreux (à
peine plus que le hasard).
\subsection{les purs/impurs}
Nous avons choisit de regarder les mots typiques utilisés, par
certains locuteurs. En effet certains mots sont transparants dans
certaines langues et sont donc plus utilisés par ceux qui la
pratiquent.

Nous avons créé trois catégories de mots pour une langue donnée :
\begin{description}
\item[les mots pures], c'est à dire les mots trouvés uniquement dans
  les textes d'une seule langue.
\item[les mots impures], qui sont des mots trouvés dans certaines
  langues (dont celle actuellement étudiée),mais pas toutes
\item[les mots globaux], qui ne sont pas présents dans la langue
  étudiée, mais qu'on peut trouver dans d'autres textes
\end{description}
À partir de là, le fait de trouver un mot pure dans un texte augmente
les chances que ce texte appartienne à la langue correspondante.
Inversement la présence de mots impures réduit les chances que le
texte appartienne à une langue donnée. 
Ce procédé nous a donné de bons résultats : $30\%$ au début puis
$51\%$ après une transformation {\color{red} En effet ,travailler qu'avec
les mots pures, le modèle est fiable qu'à 25 pourcent tandis qu'en introduisant
les globaux on augement le score à 30 pourcent. Finalement en introduisant un
fonction de minimisation des mots 1/pure, impure et globaux on obtient 50 pourcent de bonne predictions}

\subsection{les forêts aléatoires}
Pour améliorer les scores, nous avons utilisé les variables
précédement nommées aimsi que des variables générales, commes la
longueur du texte ou le nombre le mots par phrase pour nourrir un
algorithme de forêt aléatoire. Ce procédé nous a permis d'atteindre un
score de $41\%$ environ.
\section{Conclusion}
…
\end{document}
