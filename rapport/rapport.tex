\documentclass[a4paper]{article}
\usepackage[utf8]{inputenc}
\usepackage{xspace}
\usepackage[french]{babel}


\title{Rapport}
\author{ Handy-Pedro \textsc{Valery}\\
  %(valeryhandypedro@yahoo.com)
  Charles \textsc{Dehlinger}\\
  %(charles.dehlinger@universite-paris-saclay.fr)
}
\setcounter{tocdepth}{4}

\begin{document}

\maketitle

\tableofcontents

\section{introduction}
Dans ce projet, nous avons été amemés à chercher la meilleur façon de
détecter la langue maternelle d'un locuteur lorsqu'il s'exprime en
anglais. 

\section{les purs/impurs}
Nous avons choisit de regarder les mots typiques utilisés, par
certains locuteurs. En effet certains mots sont transparants dans
certaines langues et sont donc plus utilisés par ceux qui la
pratiquent.

Nous avons créé deux catégories de mots :
\begin{description}
\item[les mots pures], c'est à dire les mots trouvés uniquement dans
  les textes d'une seule langue,
\item[les mots impures], qui sont des mots trouvés dans certaines
  langues mais pas toutes.
\end{description}
À partir de là, le fait de trouver un mot pure dans un texte augmente
les chances que ce texte appartienne à la langue correspondante.
\end{document}
